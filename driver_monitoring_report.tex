\documentclass[12pt]{article}
\usepackage{geometry}
\usepackage{graphicx}
\usepackage{enumitem}
\geometry{a4paper, margin=0.7in}

\usepackage{hyperref}
\usepackage{tikz}

\begin{document}

% Title page with large margins
{\newgeometry{left=1in,right=1in,top=1.2in,bottom=1.2in}
\begin{titlepage}
    \centering
    \vspace{1cm}
    {\Large\scshape James Madison University \par}
    \vspace{0.5cm}
    {\large Information Technology (IT) Program \par}
    {\large IT 445 - Capstone Implementation \par}
    {\large Spring 2023 \par}
    \vspace{1cm}
    \rule{\textwidth}{0.5pt}
    \vspace{0.5cm}
    {\bfseries\LARGE Analyzing Autonomous Cart Passenger\\[0.2cm]
    Experience Using Emotion Detection \par}
    \vspace{0.5cm}
    \rule{\textwidth}{0.5pt}
    \vspace{0.8cm}
    {\large Monday 30\textsuperscript{th} June, 2025 \par}
    {\large 12:22:50 \par}
    \vspace{0.5cm}
    \begin{minipage}{0.3\textwidth}
        \raggedright
        \textit{Submitted to:}\\
        Dr. Samy El-Tawab
    \end{minipage}
    \begin{minipage}{0.45\textwidth}
        \raggedleft
        \textit{Authors:}\\
        Rana Moussa \par Ramez Asaad
    \end{minipage}
    \vfill
    % Optionally add logos here
    % \includegraphics[width=0.35\textwidth]{jmu_logo.png}\par
    % \vspace{1cm}
    % \includegraphics[width=0.18\textwidth]{jac_logo.png}\par
    % \vspace{1cm}
\end{titlepage}
\restoregeometry}

% Add after the title page:

\section{Overview}
This report describes a real-time driver and passenger monitoring system for autonomous carts, using computer vision and deep learning. The system detects drowsiness, yawning, and emotions from camera feeds, and provides visual feedback and warnings. The implementation is modular, robust, and designed for real-world conditions.

\section{System Scripts}
The project consists of several Python scripts, each responsible for a key part of the monitoring system:

\subsection{face\_detection.py}
This script performs real-time face detection, landmark extraction, and drowsiness/emotion analysis using a single camera feed. It demonstrates the core detection pipeline and warning logic.

\subsection{record\_side\_by\_side.py}
This script records and displays side-by-side video from two ZED stereo cameras (passenger and front/road). It overlays timestamps, detects faces and emotions, and saves the output with a unique filename. The script is modular, with functions for face mesh setup, image enhancement, and detection.

\subsection{crop\_and\_enhance\_passenger.py}
This script processes passenger camera images, cropping and enhancing them to improve detection in poor lighting. It is used for preprocessing and testing lighting robustness.

\section{System in Action}
Below are example images of the scripts running in real time. These illustrate the detection overlays, warnings, and side-by-side video output.

\begin{figure}[h!]
    \centering
    \includegraphics[width=0.8\textwidth]{example_face_detection.png}
    \caption{Face detection and drowsiness/emotion analysis (face\_detection.py)}
\end{figure}

\begin{figure}[h!]
    \centering
    \includegraphics[width=0.8\textwidth]{example_side_by_side.png}
    \caption{Side-by-side video output with overlays (record\_side\_by\_side.py)}
\end{figure}

\begin{figure}[h!]
    \centering
    \includegraphics[width=0.8\textwidth]{example_enhanced_passenger.png}
    \caption{Cropped and enhanced passenger image (crop\_and\_enhance\_passenger.py)}
\end{figure}

% Set smaller margins for the rest of the document
\geometry{a4paper, margin=0.7in}

\section{Introduction}
This project implements a real-time monitoring system for drivers and passengers in autonomous carts. Using cameras and deep learning, the system detects drowsiness, yawning, and emotions, and provides visual feedback and warnings. The system is designed to be robust, modular, and easy to use in real-world conditions.

\usetikzlibrary{shapes.geometric, arrows.meta}

\usetikzlibrary{matrix, positioning, arrows.meta}

\begin{tikzpicture}[
  block/.style={rectangle, rounded corners, minimum width=2.5cm, minimum height=1cm, text centered, draw=black, fill=blue!10},
  arrow/.style={thick,->,>=Stealth}
]
\matrix (m) [row sep=1cm, column sep=1.5cm] {
  % Row 1
  \node[block] (input) {Camera Frame Capture}; & \\
  % Row 2
  \node[block] (preprocess) {Frame Preprocessing (BGR $\rightarrow$ RGB)}; & \\
  % Row 3
  \node[block] (landmarks) {Face Detection \& Landmark Extraction (MediaPipe)}; & \node[block] (fatigue) {Fatigue \& Alertness Analysis}; \\
  % Row 4
  \node[block] (crop) {Per-Face Cropping}; & \\
  % Row 5
  \node[block] (deepface) {Emotion Analysis (DeepFace)}; & \\
  % Row 6
  \node[block] (correction) {Emotion Correction \& Smoothing}; & \\
  % Row 7
  \node[block] (output) {Visualization \& Output}; & \\
};

\draw[arrow] (input) -- (preprocess);
\draw[arrow] (preprocess) -- (landmarks);
\draw[arrow] (landmarks) -- (crop);
\draw[arrow] (crop) -- (deepface);
\draw[arrow] (deepface) -- (correction);
\draw[arrow] (correction) -- (output);
\draw[arrow] (landmarks) -- (fatigue);
\draw[arrow] (fatigue) |- (correction);
\end{tikzpicture}

\section{Project Goals}
\begin{itemize}
    \item Detect multiple faces and analyze their states independently.
    \item Recognize drowsiness, yawning, and head nodding.
    \item Detect and classify emotions in real time.
    \item Provide clear visual feedback and warnings.
\end{itemize}

\section{Technologies Used}
\begin{itemize}
    \item \textbf{OpenCV}: For real-time webcam video capture, frame processing, and image enhancement (including histogram equalization in the YUV color space).
    \item \textbf{MediaPipe}: For precise face landmark detection (eyes, mouth, nose).
    \item \textbf{DeepFace}: For emotion analysis using a deep learning model.
    \item \textbf{NumPy}: For numerical operations like distance calculations.
    \item \textbf{Python}: The overall implementation language.
    \item \textbf{Collections (deque, defaultdict)}: For maintaining per-face history and state.
    \item \textbf{ZED stereo camera}: For capturing and splitting left/right images, with 180$^\circ$ rotation to correct for upside-down orientation.
    \item \textbf{Custom Python script (\texttt{record\_side\_by\_side.py})}: For recording and displaying side-by-side video from two ZED cameras, with timestamp overlay and robust, unique naming.
\end{itemize}

\section{Core Functionality}
\subsection{Face Landmark Detection (MediaPipe)}
\begin{itemize}
    \item Tracks up to 5 faces simultaneously.
    \item Extracts key facial landmarks for eyes, mouth, and nose.
\end{itemize}
\subsection{Emotion Detection (DeepFace)}
\begin{itemize}
    \item Captures a cropped face image and predicts the dominant emotion.
    \item Applies smoothing to avoid flickering or unstable predictions.
    \item Includes logic to override fear or sad when facial context contradicts it.
    \item If the detected emotion is ``fear'' and the EAR is very low, the emotion is set to ``neutral'' to avoid false positives due to drowsiness.
\end{itemize}
\subsection{Tiredness Detection}
\begin{itemize}
    \item Eye Aspect Ratio (EAR) detects blinking or eye closure.
    \item Drowsiness detection uses a rolling average of the EAR for each detected face, smoothing out noise and reducing false positives.
    \item The rolling average window for EAR is currently set to 5 frames (configurable for further tuning).
    \item Drowsiness is flagged if the average EAR falls below a threshold (default: 0.25) for a consecutive number of frames (default: 15).
    \item Debug output prints the EAR and its rolling average for each face to assist with threshold tuning.
    \item Mouth Aspect Ratio (MAR) detects yawning.
    \item A high MAR triggers a yawning warning.
\end{itemize}
\subsection{Head Nod Detection}
\begin{itemize}
    \item Compares vertical displacement of the nose tip across frames.
    \item Large vertical shifts indicate nodding, a sign of sleepiness.
\end{itemize}
\subsection{Emotion Correction Rules}
\begin{itemize}
    \item Emotion `fear` downgraded to `neutral` if facial features are calm.
    \item Emotion `sad` reclassified if mouth corners are raised or EAR is high.
    \item Uses a rolling emotion history to ensure stability before updating UI.
\end{itemize}
\subsection{Visual Output}
\begin{itemize}
    \item Displays:
    \begin{itemize}
        \item Real-time bounding box around each face.
        \item Emotion label above the face.
        \item Warnings such as ``\textbf{⚠ Drowsiness}'', ``\textbf{⚠ Yawning}''
    \end{itemize}
\end{itemize}
\subsection{Side-by-Side Recording}
\begin{itemize}
    \item Records and displays video from two ZED cameras (passenger and front/road) side by side.
    \item Includes a timestamp overlay and robust, unique naming for output files.
    \item The script is modular, with functions for face mesh setup, cropping/enhancement, emotion detection, and front camera processing.
    \item Robust handling of multiple faces, lighting conditions, and camera feeds.
\end{itemize}

\section{Conclusion}
This project demonstrates a robust, modular, and real-time driver and passenger monitoring system for autonomous carts. By combining computer vision, deep learning, and careful image enhancement, the system reliably detects drowsiness, yawning, and emotions, providing clear visual feedback and warnings. The modular script structure and use of stereo cameras make the solution adaptable for real-world deployment and further research.

\section{Next Steps}
\begin{itemize}
    \item Further tune detection thresholds and rolling window sizes for different lighting and camera conditions.
    \item Integrate additional sensors (e.g., heart rate, steering input) for multi-modal monitoring.
    \item Expand emotion and fatigue detection to include more subtle cues and longer-term trends.
    \item Deploy the system in real vehicles and collect more data for validation and improvement.
    \item Develop a user interface for easier configuration and visualization.
\end{itemize}

\section{Sources}
\begin{itemize}
    \item OpenCV documentation: \url{https://docs.opencv.org/}
    \item MediaPipe documentation: \url{https://google.github.io/mediapipe/}
    \item DeepFace documentation: \url{https://github.com/serengil/deepface}
    \item ZED Stereo Camera: \url{https://www.stereolabs.com/zed/}
    \item Python official documentation: \url{https://docs.python.org/3/}
    \item NumPy documentation: \url{https://numpy.org/doc/}
\end{itemize}

\end{document}